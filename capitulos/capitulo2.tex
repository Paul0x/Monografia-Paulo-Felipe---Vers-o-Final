\chapter{Revisão da literatura}

\section{Robótica Móvel}
A robótica móvel é uma das grandes áreas da robótica. Robôs são, por definição, manipuladores ou mecanismos reprogramáveis e multifuncionais utilizados para mover materiais, ferramentas ou partes através de diferentes trajetórias definidas e para realização de diversas tarefas. A principal diferença entre os robôs manipuladores, amplamente aplicados em industrias, e robôs móveis é que o primeiro tipo realiza movimentos dentro de um espaço de trabalho definido e o segundo consegue se locomover no espaço, seja utilizando rodas, pés ou outro meio de locomoção. \cite{nehmzow2012mobile}
\par
Na literatura, é possível encontrar diversas classificações para os robôs móveis. Uma delas é a divisão por meio de locomoção do robô, que pode ser terrestre, aquática, aérea ou hibrida. Uma das características presentes em todas as classificações de locomoção é a ideia da biomimética, ou seja, a movimentação dos robôs é sempre inspirada na fisiologia e nos métodos de locomoção encontradas nos animais. \cite{russo2020survey} Dentro da classificação por meio de locomoção é possível realizar uma subdivisão pela forma com que esse movimento será realizada. Por exemplo, robôs terrestres podem se locomover através de pernas, rodas, peristaltismo, deslizamento, braquiação e adesão a superfícies. \cite{russo2020survey}
\par
Dentro do subgrupo de robôs móveis terrestres que utilizam rodas, também é possível realizar um agrupamento de acordo com a disposição, tração exercida e grau de mobilidade que a roda possui. As rodas podem ser do tipo mecano, uma roda omnidirecional, caster, fixas ou dirigida. O número e o tipo dessas rodas define o grau de liberdade que o robô possui. \cite{gruber2016} 
\subsection{Robô Diferencial}
Robôs diferenciais descrevem um dos possíveis arranjos existentes dentre os robôs móveis sob rodas. São robôs planares, compostos de duas rodas fixas e independentemente controláveis, normalmente dispostas de forma paralela em uma das extremidades e também uma roda do tipo caster central, que permite a locomoção em qualquer direção.
\subsubsection{Cinemática}
\section{Controle PID}
\section{Visão Computacional}
\subsection{Aquisição de Imagens}
\subsection{Processamento de Imagens}
\section{Laboratórios Remotos Didáticos}
\subsection{Tipos}
\subsection{Didática}
\subsection{Estrutura e Arquitetura}
